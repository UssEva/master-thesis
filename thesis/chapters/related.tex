% !TeX encoding = UTF-8
% !TeX spellcheck = en_GB
% !TeX root = ../thesis.tex

\chapter{Related Work}\label{ch:related_work}
There are already many papers and articles that describe the relevance of patterns and architectures that focus specifically on IoT. But besides common design patterns and frameworks, security-specific patterns for IoT applications are still in their early stages of development and documentation and therefore rare to find. Here is a short excerpt to give an overview of important publications for the IoT development and to show the areas that are still lacking in research, e.g. IoT security patterns.


\section{General Patterns for IoT Design}
Design, privacy or security patterns are only a few examples of potential solutions to common development problems that IT experts face on a daily basis. The landscape of patterns and architectures that support the IoT market is huge and research can be found in many articles like the following.

\paragraph{Applying IoT Patterns to Smart Factory Systems.} Patterns are a common way to describe abstract solutions to design problems and can be used to analyse and understand computer systems in an easier way. Reinfurt et al.~\cite{Reinfurt2017} describe in their paper IoT-specific patterns that help to design IoT systems that can be applied to the domain of smart factory systems. These patterns cover different areas and operation modes like device communication and management as well as energy supply types.

\paragraph{IoT Architectural Styles.} Other than design patterns, different architecture styles can be utilized when creating IoT systems. Muccini et al.~\cite{Muccini2018} provide a number of abstract IoT reference architectures in their paper. By conducting a systematic mapping study a set of 63 papers was chosen out of over 2,300 potential works. The results of this study will help to classify existing and future approaches for IoT styles and patterns at the architectural level.

\paragraph{Landscape of Architecture and Design Patterns for IoT Systems.} In order to get a better idea for the landscape of patterns and architectures that have accumulated over the years in research, Washizaki et al.~\cite{Washizaki2019, Washizaki2020} analysed the successes and failures of used patterns for IoT systems. Because of limited documentation and a lack of successfully executed implementations, there is a lot of room for improvement in the development of IoT-specific patterns and architectures.  


\section{Security-specific IoT Patterns}
Unlike general frameworks that can be utilized by developers to simplify their work and help to solve problems on a more abstract level, domain-specific IoT patterns that also focus on security are difficult to find. Security objectives are a huge part of every computer system to protect the user's data and identity but, unfortunately, patterns that could prevent unauthorized access are lacking. There exist many ways an attacker can compromise IoT systems that are especially easy to exploit because of a vast variety of interfaces and connections that lead to vulnerabilities. 

\paragraph{Architectural Patterns for Secure IoT Orchestrations.} Fysarakis et al.~\cite{Fysarakis2019} sketch the SEMIoTICS approach to create a pattern-driven framework that is based on already existing IoT platforms. Aiming to guarantee secure actuation and semi-automatic behaviour, the SEMIoTICS project utilizes patterns to encode dependencies between security, privacy, dependability and interoperability properties of smart objects.

\paragraph{Towards a Collection of Security and Privacy Patterns.} Organized in an hierarchical taxonomy, Papoutsakis et al.~\cite{Papoutsakis2021} collect and categorize a set of security and privacy patterns. While giving the reader an overview of security- and privacy-related objectives that are relevant in the IoT domain, the goal of this paper is to match these properties to their corresponding patterns. This usable pattern collection should guide developers to create IoT solutions that are secure and privacy-aware by design.

\paragraph{A Decade of Research on Patterns and Architectures for IoT Security.} Over the last three years, Rajmohan et al.~\cite{Rajmohan2020, iotbds20, Rajmohan2022} published different papers that review the research work regarding IoT patterns and architectures for IoT security and privacy. Although there is a rise in number of publications, there is not yet an approach of applying architectures and patterns in a way that addresses security not only on the architectural but also on the network or IoT devices level. \\

Although the IoT research in academia is thriving and there exist a variety of design patterns, architectures and frameworks that can be utilized in the development of modern IoT devices, the lack of focus on security objectives is worrisome. Security patterns can not only guide developers to build safer systems but lessen the workload by taking already existing solutions for abstract problems. This master's thesis will make an important contribution in increasing the accessibility to these already existing IoT security patterns for future work in this field.  