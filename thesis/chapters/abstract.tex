% !TeX encoding = UTF-8
% !TeX spellcheck = en_GB
% !TeX root = ../thesis.tex

\thispagestyle{plain}

\section*{Abstract}
The IoT Security Pattern Landscape: Collection and Analysis of state-of-the-art Security Patterns for IoT Development

Designing a computer system is a complex project that gets easier by utilizing the right tools. Specific patterns and architectures allow a developer to simplify problems and analyse the structure of a system. By not only applying different patterns for design but also security, the developed system can ensure a security-aware operation. But the development and usage of said security patterns is hindered by the needed expert knowledge in these areas. For this reason extensive documentation and testing in practice is needed to ensure usable patterns that can be easily applied. While design patterns for development are common for most computer systems these days, domain-specific patterns like for Internet of Things (IoT) are rare to find. Because of the different requirements that an IoT system has, existing design patterns are often not suitable for this specific use case. When IoT-specific patterns are hard to find, it is only reasonable to assume that IoT patterns that focus on security aspects are even rarer. Thus, a structured and organized catalogue of IoT security patterns is basically non-existent at this point in time. Therefore, this master's thesis proposes a systematic collection and categorization of IoT security patterns and analyses the gaps of the recent research work regarding IoT security. As a catalogue that combines 61 IoT security patterns in one place and is organized in a top to bottom approach that follows the IoT World Forum Reference Model of the IoT architecture, this collection will play an important part in future development of secure IoT solutions. After gathering the IoT security publications and analysing the created pattern catalogue, our research questions (RQ) lead to the following conclusions. RQ1 indicates that the Data Accumulation layer shows a lack of developed pattern solutions in the IoT security field. When analysing the coverage of security goals in RQ2, the availability objective falls behind the others and is only addressed by around 30\% of the total pattern collection. In RQ3 we asked ourselves which common vulnerabilities could be (partially) solved by our patterns and found that each point of the OWASP top ten is at least addressed by one security pattern. Number four among the issues on the list - the lack of ability to securely update the IoT device - was the hardest to find a solution for and is only mentioned once. To sum up, there is still a lot of room for further research in the development of IoT security patterns and expansion of the existing state-of-the-art. Therefore, we suggest a close cooperation between industry and academia to create usable patterns that help to secure our existing IoT infrastructures in the future. 