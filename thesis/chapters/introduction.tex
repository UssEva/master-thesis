% !TeX encoding = UTF-8
% !TeX spellcheck = en_GB
% !TeX root = ../thesis.tex


\chapter{Introduction}\label{ch:introduction}
Today most smart devices are highly connected in a network that allows them to interact with each other. But this level of connectivity makes it difficult even for experts to keep the overview. These intelligent networks can be quite confusing when one does not have knowledge of the right tools/frameworks or when the needed technology does not even exist, yet. Many users also don't realize how dependent one can be on these devices and what security risks a connected world can bring. The first Section~\ref{sec:def} introduces the term Internet of Things (IoT) and follows up in Section~\ref{sec:areas} with different areas of life that these technologies are likely to be found in to get familiar with IoT for the following analysis.


\section{Definition of Internet of Things}\label{sec:def}
The term IoT itself is common knowledge and these \emph{\q{things}} are well-established in our society but only a few people know what exactly the \emph{\q{Internet of Things}} is. In this section we define which devices belong to IoT and in which areas of life IoT enhances the daily routines of people.
\begin{quote}
	\emph{\q{The Internet of Things (IoT) describes the network of physical objects — things — that are embedded with sensors, software, and other technologies for the purpose of connecting and exchanging data with other devices and systems over the internet.\footnote{\href{https://www.oracle.com/in/internet-of-things/what-is-iot}{https://www.oracle.com/in/internet-of-things/what-is-iot, last accessed: 12.10.2022.}}}}
\end{quote}
Although this quote from the oracle website gives a good general definition of what IoT encompasses, it is easier to understand the idea of IoT with an example: Picture a smart home in which every gadget and appliance inside the house is connected to the internet and communicates with each other. Using your smartphone one can lock the door and get notifications if an unauthorized person enters the backyard when he is detected by the installed motion sensors. The room temperature is regulated automatically with temperature sensors that are located outside and lights are controlled according to the time of day to save energy. More common examples and different types of IoT devices are:\footnote{\href{https://learn.g2.com/iot-devices}{https://learn.g2.com/iot-devices, last accessed: 21.10.2022.}}
\begin{itemize}
	\item Home security: Motion detection, automatic recording, remote arm or disarm
	\item Industrial IoT: Security or safety systems, machine sensors, project management
	\item Healthcare and fitness: Wearable healthcare devices, activity trackers
	\item Banking and financial services: ATMs, interactive payment cards
	\item Virtual/Augmented reality: Mobile or head-mounted AR, standalone VR
\end{itemize}
But the high connectivity and continuous data exchange between devices and applications has both benefits as well as disadvantages. 

\begin{itemize}
	\item[+] Time savings. By automating activities, it saves the user a lot of time.
	\item[+] Enhanced data collection. Information is easily accessible and frequently updated in real time.
	\item[+] Safety concerns. Devices sense potential danger and warn users in time. 
\end{itemize}

\begin{itemize}
	\item[--] Security issues. Hackers may gain access to the system and steal personal information.
	\item[--] High internet dependence. Devices are unable to function effectively without the internet.
	\item[--] Complexity. There are many risks of failure in IoT devices.
\end{itemize}


\section{Application Areas}\label{sec:areas}
After we discussed what IoT is in general and which devices belong to the IoT spectrum, we look into the application areas of this technology. Therefore we divide the field of IoT into five groups\footnote{\href{https://syntegra.net/internet-of-things-the-five-types-of-iot}{https://syntegra.net/internet-of-things-the-five-types-of-iot, last accessed: 12.10.2022.}} and explore the different ways IoT can be utilized.

\paragraph{Consumer Internet of Things (CIoT).}
All kinds of appliances, gadgets and other connected devices that are meant for user consumption are a part of CIoT. Typical products include smartphones, wearables, home appliances, etc. These are in most cases connected via Wi-Fi or Bluetooth for short-range communication inside the home. 

\paragraph{Commercial Internet of Things.}
Most Commercial IoT focuses on large venues in order to improve customer experience and business conditions. As an example these IoT devices can monitor environmental conditions, manage access control systems or economize utilities in office buildings, stores, healthcare institutions, entertainment venues or hotels.

\paragraph{Industrial Internet of Things (IIoT).}
In order to improve productivity and efficiency, most large-scale factories or manufacturing plants utilize IoT systems. IIoT can also be often found in the fields of healthcare, automotive, logistics, and agriculture. 

\paragraph{Infrastructure Internet of Things.}
For the development of smart infrastructures this type of IoT tries to design IoT systems that are able to increase efficiency while decreasing its costs. These IoT infrastructures are then used in monitoring and controlling operations that deal with public infrastructures like bridges, railway tracks or even windfarms.

\paragraph{Internet of Military Things (IoMT).}
Like its name suggests, IoMT or also called Internet of Battlefield Things (IoBT) is an important part of the modern military. By interconnecting ships, planes, tanks, soldiers and even drones on the battlefield, not only the situational awareness during war, but also risk assessments and response times can be greatly improved. The data that is collected during monitoring can then be utilized to enhance the IoT systems, equipment, military practices and strategies. 
